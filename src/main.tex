%! Author = Marjan
%! Date = 11/09/2025

% Preamble
\documentclass[a4paper, 11pt]{book}

% Packages
\usepackage{amsmath}
\usepackage{geometry}
\usepackage{fancyhdr}
\usepackage{hyperref}
\usepackage[framemethod=tikz]{mdframed}
\usepackage{tikz}
\usepackage[utf8]{inputenc}
\usepackage{sublabel}
%\usepackage{beamerbasenotes}
%\usepackage[toc]{glossaries} % 'toc' adds the glossary to the table of contents
%\makeglossaries
\usetikzlibrary{positioning, arrows.meta, shapes.multipart}
\renewcommand{\rmdefault}{phv} % Set the default font family to Helvetica
% ---- Set page margins (A4 paper)
\geometry{top=1in, bottom=1in, left=1in, right=1in}

% ---- Set up the header
\pagestyle{fancy}
\fancyhf{}
\fancyhead[L]{Machine Learning Notes}
%\fancyhead[C]{Your Name}
\fancyhead[R]{\thepage}

% ---- For Notes
\newmdenv[
    topline=false,
    bottomline=false,
    rightline=false,
    innerrightmargin=0pt
]{siderule}

\newenvironment{note}{
    \begin{siderule}
        \textbf{Note: }
        }{
    \end{siderule}}

% Document
\begin{document}

    % Title Page
    \begin{titlepage}
        \centering
        \vspace*{2in}
        \Huge \textbf{Data Architecture Notes}
%        \vfill
%        \Large Your Name
%        \vfill
%        \Large Date: \today
    \end{titlepage}

    \setcounter{section}{0}

    \newpage

    \tableofcontents
    \newpage

    \listoffigures
    \newpage

    Python:
    https://scikit-learn.org/stable/

    Scala
    https://spark.apache.org/mllib/


    \part{Mathematics for Machine Learning} % Pick one
    \section{Linear Algebra}
    \section{Calculus}
    \section{Probability and Statistics}


    \part{Mathematics for Machine Learning}
    \section{Mathematics for Machine Learning: Linear Algebra}
    \section{Mathematics for Machine Learning: Multivariate Calculus}
    \section{Mathematics for Machine Learning: PCA}
    \subsection{Statistics of Datasets}
    \subsection{Inner Products}
    \subsection{Orthogonal Projects}
    \subsection{Principal Components Analysis}

    \part{Linear Algebra for Machine Learning and Data Science}
    \section{Systems of Linear equations}
    \section{Solving systems of linear equations}
    \section{Vectors and Linear Transformations}ó
    \section{Determinants and Eigenvectors}

    \part{Machine Learning Specialisation}

    \section{Supervised Machine Learning: Regression and Classification}
    \subsection{Overview of machine learning}
    \subsection{Supervised vs Unsupervised Machine Learning}
    \subsection{Regression model}
    \subsection{Train the model with gradient descent}
    \subsection{Multiple Linear Regression}
    \subsection{Gradient descent in practice}
    \subsection{Classification with logistic regression}
    \subsection{Cost function with logistic regression}
    \subsection{Gradient descent for logistic regression}
    \subsection{The problem overfitting}

    \section{Advanced Learning Algorithms}
    \subsection{Neural networks intuition}
    \subsection{Neural network model}
    \subsection{TensorFlow implementation}
    \subsection{Neural network implementation in Python}
    \subsection{Speculations on artificial general intelligence (AGI)}
    \subsection{Vectorisation (Optional)}
    \subsection{Neural Network Training}
    \subsection{Activation Functions}
    \subsection{Multiclass Classification}
    \subsection{Additional Neural Network Concepts}
    \subsection{Back Propagation}
    \subsection{Advice for applying machine learning}
    \subsection{Advice for applying machine learninig}
    \subsection{Bias and variance}
    \subsection{Machine learning development process}
    \subsection{Skewed datasets}
    \subsection{Decision Trees}
    \subsection{Decision Tree Learning}
    \subsection{Tree ensembles}

    \section{Unsupervised Learning, Recommenders, Reinforcement Learning}
    \subsection{Clustering}
    \subsection{Anomaly Detection}
    \subsection{Collaborative filtering}
    \subsection{Recommender systems implementation details}
    \subsection{Content-based filtering}
    \subsection{Principal Component Analysis}
    \subsection{Reinforcement Learning introduction}
    \subsection{State-action value function}
    \subsection{Continuous state spaces}

    \part{Deep learning specialisation}

    \section{Neural Networks and Deep Learning}
    \subsection{What is deep learning?}
    \subsection{Logistic Regression as a Neural Network}
    \subsection{Python and Vectorisation}
    \subsection{Shallow Neural Network}
    \subsection{Deep Neural Network}

    \section{Improving Deep Neural Networks: Hyperparameter Tuning, Regularization and Optimization}
    \subsection{Structuring Machine Learning Projects}
    \subsection{Setting up a Machine Learning Application}
    \subsection{Regularising a Neural Network}
    \subsection{Setting up a Optimisation Problem}
    \subsection{Optimisation Algorithms}
    \subsection{Hyperparameter Tuning}
    \subsection{Batch Normalisation}
    \subsection{Multi-Class Classification}
    \subsection{Introduction to Programming Frameworks}

    \section{Convolution Neural Networks}
    \subsection{Convolutional Neural Networks}
    \subsection{Deep Convolutional Models: Case Studies}
    \subsection{Detection Algorithms}
    \subsection{Face Recognition}
    \subsection{Neural Style Transfer}

    \section{Sequence Models}
    \subsection{Recurrent Neural Networks}
    \subsection{Introduction to Word Embeddings}
    \subsection{Learning Word Embeddings: Word2vec and GloVe}
    \subsection{Applications using Word Embeddings}
    \subsection{Sequence Models and Attention Mechanism}
    \subsection{Transformer Network}


    \part{Generative Adverserial Networks (GANs) Specialisation}

    \section{Build Basic Generative Adversarial Networks}
    \subsection{Intro To GANs}
    \subsection{Deep Convolutional GANs}
    \subsection{Wasserstein GANs with Gradient Penalty}
    \subsection{Conditional GAN and Controllable Generation}

    \section{Build Better Generative Adversarial Networks}
    \subsection{Evaluation of GANs}
    \subsection{GAN Disadvantages and Bias}
    \subsection{StyleGAN and Advancements}

    \section{Apply Generative Adverserial Networks}
    \subsection{GANs for Data Augmentation and Privacy}
    \subsection{Image-to-Image Translation with Pix2Pix}
    \subsection{Unpaired Translation with CycleGAN}

    \part{Natural Language Processing Specialisation}

    \section{Natural Language Processing with Classification and Vector Spaces}
    \subsection{Sentiment Analysis with Logistic Regression}
    \subsection{Sentiment Analysis with Naive Bayes}
    \subsection{Vector Space Models}
    \subsection{Machine Translation and Document Search}

    \section{Natural Language Processing with Probabilitsic Models}
    \subsection{Autocorrect}
    \subsection{Part of Speech Tagging and Hidden Markov Models}
    \subsection{Autocomplete and Language Models}
    \subsection{Word embeddings with neural networks}
    
    \section{Natural Language Processing with Sequence Models}
    \subsection{Recurrent Neural Network for Language Modeling}
    \subsection{LSTMs and Named Entity Recognition}
    \subsection{Siamese Networks}
    
    \section{Natural Language Processing with Attention Models}
    \subsection{Neural Machine Translation}
    \subsection{Text Summarisation}
    \subsection{Question Answering}

    \part{AI for Scientific Research Specialisation}

    \section{Introduction to Data Science and scikit-learn in Python}
    \subsection{Introduction to Python Programming for Hypothesis Testing}
    \subsection{Creating a Hypothesis: Numpy, Pandas and Scikit-Learn}
    \subsection{Scikit-Learn: ML for Hypothesis Testing}
    \subsection{Using Classification to Predict the Presence of Heart Disease}
    
    \section{Machine Learning Models in Science}
    \subsection{Before the AI: Preparing and Preprocessing Data}
    \subsection{Foundational AI Algorithms: K-Means and SVM}
    \subsection{Advanved AI: Neural Netoworks and Decision Trees}
    
    \section{Neural Networks and Random Forests}
    \subsection{Introduction to Neural Networks}
    \subsection{Deep Dive into Neural Networks}
    \subsection{Exploring Random Forests}

    \section{Project}

\end{document}